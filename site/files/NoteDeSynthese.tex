\documentclass[a4paper,11pt]{article}
\usepackage[T1]{fontenc}
\usepackage[utf8]{inputenc}

\usepackage{amsfonts}
\usepackage{amsmath}
\usepackage{amssymb}

\newcommand{\tfjm}{$\mathbb{TFJM}^2$}

% ATTENTION :
% LA NOTE DE SYNTHÈSE DOIT ÊTRE ENVOYÉE ENVOYER PAR EMAIL À L'ADRESSE syntheses@tfjm.org
% LE FICHIER DOIT ÊTRE AU FORMAT PDF ET NE PEUT PAS FAIRE PLUS DE DEUX PAGES
% LE NOM DU FICHIER DOIT ÊTRE T-P-N-XYZ.pdf AVEC
% — T LE NUMÉRO DU TOUR
% — P LA POULE DE L'ÉQUIPE
% — N LE NUMÉRO DE PROBLÈME
% — XYZ LE TRIGRAMME DE L'ÉQUIPE

\textwidth 173mm \textheight 235mm \topmargin -50pt \oddsidemargin
-0.45cm \evensidemargin -0.45cm

\title{
\Huge \textsc{
\tfjm \\
\LARGE Note de synthèse}}
\date{}

\pagestyle{empty}

\begin{document}

\maketitle
\thispagestyle{empty}

\vspace{-1cm}
Tour $N$, Poule $M$, Problème $n$, d\'efendu par l'\'equipe {\sc Nom-de-l'\'equipe-du-D\'efenseur}.\\

Synth\`ese par l'\'equipe {\sc Nom-de-votre-\'equipe} dans le r\^ole de {\sc Votre-r\^ole (Rapp/Opp/Obs)}.

\section*{Compte-rendu de la solution}

En quelques lignes, on donnera un bref résumé du contenu de la solution qui précise les questions qui ont été correctement traitées ainsi que les questions non résolues, ainsi que toute remarque pouvant être utile pour évaluer la solution.

\section*{Erreurs et imprécisions}

Il s'agit de la partie la plus importante du rapport. Il est conseillé d'établir deux listes (avec des références précises à la solution) contenant toutes les erreurs trouvées dans la solution de l'équipe présentant le problème. La première rend compte des erreurs les plus importantes, et la seconde contient les erreurs qui ont moins de conséquences. Les deux listes devraient être classées par ordre décroissant d'importance. Il n'est pas nécessaire de préciser les erreurs typographiques ainsi que les détails mineurs.

En cas d'erreur majeure, on pourra expliquer rapidement comment corriger l'erreur (sans rentrer dans les détails).


\section*{Remarques formelles}

On pourra donner son avis concernant la forme de la solution, en particulier si la présentation nuit à la compréhension de la solution. Cette partie est facultative.


\section*{\'Evaluation qualititative de la solution}

On donnera ici son avis concernant la solution. On mettra en valeur les points positifs de la solution (par exemple des idées importantes, originales, etc.) et précisera ce qui aurait pu rendre la solution meilleure (éviter les erreurs éventuellement mentionnées précédement, utiliser des idées présentes pour résoudre d'autres questions, étudier des généralisations, etc.)


\textbf{\'Evaluation.} On donnera un adjectif résumant la qualité de la solution: excellente, bonne, suffisante, passable. Cet adjectif doit être dûment motivé par les précédentes parties.

\end{document} 